\documentclass{article}			% Try book, report etc.

\usepackage[utf8]{inputenc}		% Character encoding
\usepackage[T1]{fontenc}			% Font encoding
\usepackage[english]{babel}		% Language setting
\usepackage{enumerate}			% Custom enumeration lists
\usepackage{shortlst}			% Compact inline listings
\usepackage{natbib}				% Harvard-style bibliography with several options
\usepackage{float}				% Custom table and figure alignment
\usepackage{multirow}			% Multiple rows in a table
\usepackage{hyperref}			% References
\usepackage{makecell}			% Custom table lines
\hypersetup{						% Setup for this package
    colorlinks,
    citecolor=black,
    filecolor=black,
    linkcolor=black,
    urlcolor=blue
}

\usepackage{amsmath,mathtools}	% AMS math package, with bugfixes in mathtools
\usepackage{amssymb}				% Math symbols
\usepackage{amsthm}				% Custom theorem environments
\usepackage{units}				% Numerical fractions

% Custom theorems
% \newtheorem{command}[counter]{Display name}
\newtheorem{theorem}{Theorem}[]
\newtheorem{acknowledgement}[theorem]{Acknowledgment}
\newtheorem{algorithm}[theorem]{Algorithm}
\newtheorem{axiom}[theorem]{Axiom}
\newtheorem{case}[theorem]{Case}
\newtheorem{claim}[theorem]{Claim}
\newtheorem{conclusion}[theorem]{Conclusion}
\newtheorem{condition}[theorem]{Condition}
\newtheorem{conjecture}[theorem]{Conjecture}
\newtheorem{corollary}[theorem]{Corollary}
\newtheorem{criterion}[theorem]{Criterion}
\newtheorem{definition}[theorem]{Definition}
\newtheorem{exercise}[theorem]{Exercise}
\newtheorem{lemma}[theorem]{Lemma}
\newtheorem{notation}[theorem]{Notation}
\theoremstyle{definition}
\newtheorem{problem}[theorem]{Problem}
\newtheorem{assumption}[theorem]{Assumption}
\newtheorem{proposition}[theorem]{Proposition}
\theoremstyle{remark}
\newtheorem{example}[theorem]{Example}
\newtheorem{remark}[theorem]{Remark}
\newtheorem{solution}[theorem]{Solution}
\newtheorem{summary}[theorem]{Summary}

\renewcommand{\qedsymbol}{$\blacksquare$}

\title{Math in \LaTeX}			% This is how the symbol for LaTeX and TeX are typeset
\author{Attila Gyetvai}

%%%%%%%%%%%%%%%%%%%%%%%%%%%%%%%%%

\begin{document}

\maketitle						% If date is not specified, current date is included; for no 										  date, write \date{}

\section{Environments, equations}

This is an inline math environment: $f(x) = 2 x$.

This is a dedicated math environment:
\[
f(x) = 2 x
\]
You can do it like this too\footnote{However, it is fragile, so you should avoid it.}:
$$
f(x) = 2 x
$$
The best way to write dedicated math expressions is the equation environment:
\begin{equation}
f(x) = 2 x
\end{equation}
For no numbering, use the starred version:
\begin{equation*}
f(x) = 2 x
\end{equation*}
Systems of equations are put in the align environment:
\begin{align*}
f(x) &= 2 x \\
g(x) &= 9 \, x			% Note that there is no line break before intertext!
\intertext{For short texts in between:}
h(x) &= 0.1 x
\end{align*}

\section{Formulae, symbols}

\begin{align*}
a^2 + b^2 &= c^2 \\
a_1^2 + a_2^2 &= a_3^2 \\
a_{11}^2 + a_{12}^2 &= a_{13}^2 \\		% If multiple characters in super- or subscript, use 										  braces!
\sin x = \alpha &\iff x = \arcsin x \\
\Pi &\implies \pi \\
\frac{2 x}{4 y} &= \frac{x}{2 y} \\
\nicefrac{1}{2} &\neq \mathbb{N} \\
\log x \to 0 \mbox{ as } x \to - \infty &\iff \lim_{x \to - \infty} x = 0
\end{align*}

\section{Operators}

\begin{align*}
y &= \beta_0 + \beta_1 \, x_1 + u \\
y &= \beta_0 + \beta_1 \, x_1 + \beta_2 \, x_1^2 + u \\
y &= \beta_0 + \sum_{i=1}^j \beta_i \, x_i + u \\
\hat{y} &= \hat{\beta}_0 + \sum_{i=1}^j \hat{\beta}_i \, x_i \\	% Hat
\widehat{educ} &= \hat{\beta}_0 + \sum_{i=1}^j \hat{\beta}_i \, x_i \\	% Wide hat
\bar{y} &= \beta_0 + \sum_{i=1}^j \beta_i \, \bar{x}_i + u \\		% Bar
\end{align*}
Inline version: $y = \beta_0 + \sum_{i=1}^j \beta_i \, x_i + u$.
Inline version with full typography: $\displaystyle y = \beta_0 + \sum_{i=1}^j \beta_i \, x_i + u$.

\section{Vectors, matrices}

\begin{align*}
\begin{matrix}
1 & 2
\end{matrix} \qquad
\begin{matrix}
1 \\
2
\end{matrix}\qquad
\begin{matrix}
1 & 2 \\
1 & 2
\end{matrix} \\
\begin{pmatrix}
1 & 2
\end{pmatrix} \qquad
\begin{bmatrix}
1 \\
2
\end{bmatrix}\qquad
\begin{vmatrix}
1 & 2 \\
1 & 2
\end{vmatrix}
\end{align*}

\subsection*{All together}

\begin{align*}
\begin{pmatrix}
\alpha_{11} & \alpha_{12} \\
\alpha_{21} & \alpha_{22}
\end{pmatrix}
\begin{pmatrix}
x_1 \\
x_2
\end{pmatrix}
\iff
\sum_{i=1}^2 \alpha_{i1} \, x_1 + \sum_{i=1}^2 \alpha_{i2} \, x_2
\end{align*}

\section{Theorems}

\begin{theorem}
All markers are blue.
\end{theorem}
\begin{proof}
By counterexample.
You can buy black ones in the store.
\end{proof}
\begin{problem}[Varian 32.116]
Two guys, $x$ and $y$ play the chicken game.
\end{problem}
\setcounter{theorem}{0}
\begin{assumption}
There might be pink markers.
\end{assumption}
\addtocounter{theorem}{1}
\begin{summary}
Theorems in \LaTeX\ are cool.
\end{summary}

\section{Regression outputs}

\subsection{Equation}

\begin{align*}
\widehat{wage} &= - \underset{(.77)}{3.39} + \underset{(.0538)}{.6443} \, educ + \underset{(.0110)}{.0701} \, exper \\
n &= 526, \quad R^2 = .2252
\end{align*}

\subsection{Table}

\begin{table}[H]
\centering
\begin{tabular}{cr@{.}lr@{.}lr@{.}lr@{.}lr@{.}lr@{.}l}
\hline 
$wage$ & \multicolumn{2}{c}{Coeff.} & \multicolumn{2}{c}{St.e.} & \multicolumn{2}{c}{$t$-stat.} & \multicolumn{2}{c}{$p$-value} & \multicolumn{4}{c}{95\% conf. int.} \\ 
\hline 
$educ$   &   &6443 & &0538 &  11&97 & &000 &   &5386 &   &7500 \\ 
$exper$  &   &0701 & &0110 &  6&39  & &000 &   &0485 &   &0917 \\ 
constant & -3&39   & &77   & -4&42  & &000 & -4&90   & -1&88 \\
\hline
\multicolumn{13}{c}{$n=526, \quad R^2 = .2252$}
\end{tabular} 
\end{table}

\subsection*{My way}

\begin{table}[H]
\centering
\begin{tabular}{cccr@{.}lc}
\Xhline{3\arrayrulewidth}
\multicolumn{6}{c}{Dependent variable: $wage$} \\
& & {\small Coefficient} & \multicolumn{2}{c}{\small $t$-stat.} & \\
\cline{2-5}
& $educ$ & $\underset{(.0538)}{.6443}^{***}$ & 11&97 & \\
& $exper$ & $\underset{(.0110)}{.0701}^{***}$ & 6&39 & \\
& constant & $- \underset{(.77)}{3.39}^{***}$ & -4&42 & \\
\cline{2-5}
\multicolumn{6}{c}{$n=526 \qquad \bar{R}^2 = .2222$} \\
\Xhline{3\arrayrulewidth}
\end{tabular}
\caption{OLS regression. 
{\scriptsize Standard errors in parentheses.
$^{***}$: significance at the 1\% level.}}
\end{table}

\end{document}