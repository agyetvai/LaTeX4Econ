\documentclass[11pt,a4paper]{article}	% Define document class

\usepackage[utf8]{inputenc}				%! Character encoding
\usepackage[T1]{fontenc}					%! Font encoding
\usepackage[english]{babel}				%! Language setting
\usepackage[margin=2.5cm]{geometry}		% Margin size
\usepackage{enumerate}					% Custom enumeration lists
\usepackage{shortlst}					% Inline listings
\usepackage{natbib}						% Harvard-style bibliography with several options
\usepackage{float}						% Custom table and figure alignment
\usepackage{multirow}					% Multiple rows in a table
\usepackage{makecell}					% Custom table lines
\usepackage{hyperref}					% References

\hypersetup{								% Setup for hyperref package
    colorlinks,
    citecolor=black,
    filecolor=black,
    linkcolor=black,
    urlcolor=blue
}

\title{Description of Economics}
\author{Attila Gyetvai}
\date{}									% Leave empty for no date

%%%%%%%%%%%%%%%%%%%%%%%%%%%%%%%%%

\begin{document}

\maketitle

\begin{abstract}
This is a short article about economics.\footnote{Most probably bullshit.}
The whole text is copied from \href{http://en.wikipedia.org/wiki/Economics}{Wikipedia}.
You can contact the author at \href{mailto:Gyetvai_Attila@student.ceu.hu}{this email address}.
\end{abstract}

\tableofcontents

\section{Intro}\label{sec:intro}

Economics is the social science that \textbf{analyzes} the production, distribution, and consumption of goods and services. \cite{wikiecon}
The term economics comes from the Ancient Greek \emph{oikonomia} (``management of a household, administration'') from \emph{oikos} (``house'') + \emph{nomos} (``custom'' or ``law''), hence ``rules of the house(hold)''.
Political economy was the earlier name for the subject, but economists in the late 19th century suggested ``economics'' as a shorter term for ``economic science'' that also avoided a narrow political-interest connotation and as similar in form to ``mathematics'', ``ethics'', and so forth.

\section{Focus}\label{sec:focus}

A focus of the subject is how economic agents behave or interact and how economies work.
Consistent with this, a primary textbook distinction is between microeconomics and macroeconomics.
\subsection*{Micro}
Microeconomics examines the behavior of basic elements in the economy, including individual agents (such as households and firms or as buyers and sellers) and markets, and their interactions.
\subsection*{Macro}
Macroeconomics analyzes the entire economy and issues affecting it, including unemployment, inflation, economic growth, and monetary and fiscal policy.

\section*{Other distinctions}

Other broad distinctions include those between positive economics (describing ``what is'') and normative economics (advocating ``what ought to be''); between economic theory and applied economics; between rational and behavioral economics; and between mainstream economics (more ``orthodox'' and dealing with the ``rationality-individualism-equilibrium nexus'') and heterodox economics (more ``radical'' and dealing with the ``institutions-history-social structure nexus'').
For more, see tables \ref{tab:econ-focus} and \ref{tab:detail}.

Economic analysis may be applied throughout society, as in business, finance, health care, and government, but also to such diverse subjects as crime, education, the family, law, politics, religion, social institutions, war, and science.
At the turn of the 21st century, the expanding domain of economics in the social sciences has been described as economic imperialism.

\begin{table}[H]
\centering
\begin{tabular}{|c|c|c|}
\hline 
 & Focus 1 & Focus 2 \\ 
\hline 
Micro & individual agents & interactions \\ 
Macro & entire economy & issues \\
\hline 
\end{tabular}
\caption{Focuses in economics.}
\label{tab:econ-focus}
\end{table}

\section{More sophisticated tables}

\begin{table}[H]
\centering
\begin{tabular}{|c|cc|}
\cline{2-3}
\multicolumn{1}{c}{} & \multicolumn{2}{|c|}{Focuses} \\ 
\hline 
\multirow{2}{*}{Micro} & individual agents & interactions \\
& \multicolumn{2}{l|}{blabla} \\
\hline
\multirow{2}{*}{Macro} & entire economy & issues \\
& \multicolumn{2}{r|}{blabla} \\
\hline 
\end{tabular}
\caption{More details.}
\label{tab:detail}
\end{table}

\begin{table}[H]
\centering
\begin{tabular}{cr@{.}l}
\Xhline{3\arrayrulewidth}
pi & 3 & 14 \\
\Xhline{3\arrayrulewidth}
\end{tabular}
\end{table}

\bibliographystyle{apalike}		% Define the exact style
\bibliography{mybib}				% Type the chosen filename

\end{document}