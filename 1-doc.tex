\documentclass[11pt]{article}			%! Try book, report etc.

\usepackage[utf8]{inputenc}		%! Character encoding
\usepackage[T1]{fontenc}			%! Font encoding
\usepackage[english]{babel}		%! Language setting
\usepackage{enumerate}			% Custom enumeration lists
\usepackage{shortlst}			% Compact inline listings

\title{Description of Economics}
\author{Attila Gyetvai}

%%%%%%%%%%%%%%%%%%%%%%%%%%%%%%%%%

\begin{document}

\maketitle						% If date is not specified, current date is included; for no 								  date, write \date{}

\begin{abstract}
	This is a short article about economics.
\end{abstract}

Economics is the social science that \textbf{analyzes} the production, distribution, and consumption of goods and services.
The term economics comes from the Ancient Greek \emph{oikonomia} (``management of a household, administration'') from \emph{oikos} (``house'') + \emph{nomos} (``custom'' or ``law''), hence ``rules of the house(hold)''.
Political economy was the earlier name for the subject, but economists in the late 19th century suggested ``economics'' as a shorter term for ``economic science'' that also avoided a narrow political-interest connotation and as similar in form to ``mathematics'', ``ethics'', and so forth.

A focus of the subject is how economic agents behave or interact and how economies work.
Consistent with this, a primary textbook distinction is between microeconomics and macroeconomics.
Microeconomics examines the behavior of basic elements in the economy, including individual agents (such as households and firms or as buyers and sellers) and markets, and their interactions.
Macroeconomics analyzes the entire economy and issues affecting it, including unemployment, inflation, economic growth, and monetary and fiscal policy.

\smallskip

Other broad distinctions include those between positive economics (describing ``what is'') and normative economics (advocating ``what ought to be''); between economic theory and applied economics; between rational and behavioral economics; and between mainstream economics (more ``orthodox'' and dealing with the ``rationality-individualism-equilibrium nexus'') and heterodox economics (more ``radical'' and dealing with the ``institutions-history-social structure nexus'').

\bigskip

\begin{flushright}
	Economic analysis may be applied throughout society, as in business, finance, health care, and government, but also to such diverse subjects as crime, education, the family, law, politics, religion, social institutions, war, and science.
	At the turn of the 21st century, the expanding domain of economics in the social sciences has been described as economic imperialism.
\end{flushright}

\newpage

\begin{itemize}
	\item
	oikos
	\item
	nomos
	\begin{itemize}
		\item
		'tis nomos
	\end{itemize}
\end{itemize}

\hspace*{2em}

\begin{enumerate}[(a)]
	\item
	oikos
	\item
	nomos
\end{enumerate}

You also can use inline enumerations with the shortlst package:

\begin{runenumerate}
	\item
	Just
	\item
	like
	\item
	\hfill this.
\end{runenumerate}

\end{document}